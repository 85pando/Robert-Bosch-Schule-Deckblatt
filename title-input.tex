% Diese Datei dient dazu, die einzelnen Punkte des Titelblattes festzulegen.

% Titel der Arbeit
\newcommand{\MyTitle}{Titel der Arbeit, kann durchaus auch grad ein etwas l\"angerer Titel sein\dots}
% Name des Autors
\newcommand{\MyAuthor}{<Autor>n}
% Klasse des Autors
\newcommand{\MyKlasse}{<Klassenbezeichnung>}
% Kontaktaddresse (Email) des Autors
\newcommand{\MyKontaktaddresse}{<email>}

% Name der Schule
\newcommand{\MySchuleA}{Robert-Bosch-Schule Ulm}
\newcommand{\MySchuleB}{Fachschule f\"ur Technik}
\newcommand{\MyRichtung}{Fachrichtung Maschinenbau}

% Name und Anschrift der Firma
\newcommand{\MyFirma}{<Firma>}
\newcommand{\MyAbteilung}{<Abteilung>}
\newcommand{\MyFirmenStrasse}{<Firmensstrasse>}
\newcommand{\MyFirmenOrt}{<PLZ, Ort>}

% Abgabedatum
\newcommand{\MyDatum}{<TT.MM.JJJJ>}
% Schuljahr
\newcommand{\MySchuljahr}{<JJJJ / JJJJ>}

%Daten von Betreuer A
\newcommand{\MyBetreuerA}{<Vorname1 Name1>}
\newcommand{\MyTitelA}{<Titel1>}
\newcommand{\MyEmailA}{<Email1>}
\newcommand{\MyTeleA}{<Telefon1>}

%Daten von Betreuer B
\newcommand{\MyBetreuerB}{<Vorname2 Name2>}
\newcommand{\MyTitelB}{<Titel2>}
\newcommand{\MyEmailB}{<Email2>}
\newcommand{\MyTeleB}{<Telefon2>}